% LaTeX Curriculum Vitae Template
%
% Copyright (C) 2004-2009 Jason Blevins <jrblevin@sdf.lonestar.org>
% http://jblevins.org/projects/cv-template/
%
% You may use use this document as a template to create your own CV
% and you may redistribute the source code freely. No attribution is
% required in any resulting documents. I do ask that you please leave
% this notice and the above URL in the source code if you choose to
% redistribute this file.

\documentclass[letterpaper]{article}

\usepackage{hyperref}
\usepackage{geometry}
\usepackage{import} % To import email.
\usepackage{marvosym} % face package
\usepackage{xcolor,color}
 \usepackage{fontawesome}

% Comment the following lines to use the default Computer Modern font
% instead of the Palatino font provided by the mathpazo package.
% Remove the 'osf' bit if you don't like the old style figures.
\usepackage[T1]{fontenc}
\usepackage[sc,osf]{mathpazo}

% Set your name here
\def\name{POLC-2300 Introduction to Comparative Politics}

% Replace this with a link to your CV if you like, or set it empty
% (as in \def\footerlink{}) to remove the link in the footer:
\def\footerlink{}
% \href{http://www.hectorbahamonde.com}{www.HectorBahamonde.com}

% The following metadata will show up in the PDF properties
\hypersetup{
  colorlinks = true,
  urlcolor = blue,
  pdfauthor = {\name},
  pdfkeywords = {political science, comparative politics},
  pdftitle = {\name: Syllabus},
  pdfsubject = {Syllabus},
  pdfpagemode = UseNone
}

\geometry{
  body={6.5in, 8.5in},
  left=1.0in,
  top=1.25in
}

% Customize page headers
\pagestyle{myheadings}
\markright{{\tiny \name}}
\thispagestyle{empty}

% Custom section fonts
\usepackage{sectsty}
\sectionfont{\rmfamily\mdseries\Large}
\subsectionfont{\rmfamily\mdseries\itshape\large}

% Don't indent paragraphs.
\setlength\parindent{0em}

% Make lists without bullets
\renewenvironment{itemize}{
  \begin{list}{}{
    \setlength{\leftmargin}{1.5em}
  }
}{
  \end{list}
}


% email input begin
\newread\fid
\newcommand{\readfile}[1]% #1 = filename
{\bgroup
  \endlinechar=-1
  \openin\fid=#1
  \read\fid to\filetext
  \loop\ifx\empty\filetext\relax% skip over comments
    \read\fid to\filetext
  \repeat
  \closein\fid
  \global\let\filetext=\filetext
\egroup}
\readfile{/Users/hectorbahamonde/RU/Bibliografia_PoliSci/email.txt}
% email input end





\begin{document}

% Place name at left
%{\huge \name}

% Alternatively, print name centered and bold:
\centerline{\huge \bf \name}

\vspace{0.25in}

\begin{minipage}{0.45\linewidth}
  Tulane University \\
  Center for Inter-American Policy \& Research \\
  Richardson Building, \\
  Second Floor, Room M \\
  New Orleans, LA 70112\\
  \\
  \\

\end{minipage}
\hspace{4cm}\begin{minipage}{0.45\linewidth}
  \begin{tabular}{ll}
{\bf Last updated}: \today. \\
 {\bf Download last version} \href{https://github.com/hbahamonde/Comparative_Politics_UGRAD/raw/master/Bahamonde_Comparative_Politics_Syllabus_UGRAD.pdf}{here}.\\
  % {\bf {\color{red}{\scriptsize Not intended as a definitive version}}} %\\
    \\
    \\
    \\
    \\
    \\
  \end{tabular}
\end{minipage}

\vspace{-5mm}
{\bf Professor}: Hector Bahamonde\\
%\texttt{e:}\href{mailto:hbahamonde@tulane.edu}{\texttt{hbahamonde@tulane.edu}}\\
\texttt{e:}\href{mailto:\filetext}{\texttt{\filetext}}\\
\texttt{w:}\href{http://www.hectorbahamonde.com}{\texttt{www.hectorbahamonde.com}}\\
{\bf Class meetings}: \texttt{MWF} 8:00-8:50.\\
{\bf Location}: Stanley Thomas Hall 302.\\
{\bf Office Hours}: Make an appointment \href{https://calendly.com/bahamonde/officehours}{\texttt{here}}.\\
{\bf Class Website and Materials}: \href{https://tulane.instructure.com/courses/2170180}{\texttt{Canvas}}.

\subsection*{Overview and Objectives}

This {\bf {\color{blue}undergraduate-level course}} offers an introduction to core concepts and theories in comparative politics subfield. The course is divided into four parts: (1) An introduction to the subfield  (substantively and methodologically); (2) the state, its origins and consequences; (3) democracy, dictatorship, regime change, electoral politics; (4) political development and colonialism.
\\
\\
During the semester, we will focus on a number of very interesting questions. \emph{Why is the state so (in)effective? Why are some societies more violent than others? What can we learn by comparing different electoral systems? Is religion (or another form of `culture') responsible for explaining democratic failures? Do diverse societies `do better' than cohesive societies? What can we learn by `comparing' countries, elections, events, economies or political leaders?} These and other questions are still subject of great debate in comparative politics. The papers and chapters will draw from what we call `the core' that defines our subfield. Comparative politics is both a \emph{substantive} as well as a \emph{methodological} area of research. That is, we are not only interested in \emph{what} is happening/has happened, but also in \emph{how} we learn and define those things. You will quickly realize that `concepts' are fundamental. For example, we are still debating what a `democracy' is since we don't agree on what are the constitutive elements that define what a `democracy' is. Well, we will spend some time talking about some cases and also discussing some important methodological issues. You will also quickly realize that comparative politics is a very \emph{flexible} subfield. Any country is of interest for us. Single-cases as well as regional approaches (i.e. `Africa,' `Latin America,' etc.) are acceptable. To convince you of that, I've marked with a ``{\color{brown}\faGlobe}'' symbol the regions/countries where we're going to be ``travelling'' to. A number of methodologies, e.g. quantitative \emph{and} qualitative approaches are possible. Any time period, and (almost any) topic, are interesting for us: from the rise of the Babylonian state, to the exit of the United Kingdom from the European Union. And such, we comparativists, borrow from sociology, economics, history, political theory, among others. 
\\
\\
I hope this course catches your attention, in the expectation that you continue taking more comparative politics courses. Most of all, I hope you see what a diverse world, practices (informal and formal) we have. {\bf Welcome!}


\subsection*{Course Learning Objectives}
 
Upon successful completion of this course, you will be able to:

\begin{itemize}
	\item[$\bullet$] Acquire an understanding of the main comparative politics theories and topics.
	\item[$\bullet$] Use the comparative method and analysis in the political science literature.
	\item[$\bullet$] Consume `critically' the comparative politics literature.
\end{itemize}


\subsection*{Classroom Etiquette}
 

\begin{itemize}
	\item[$\bullet$] Please, do not eat during class. Beverages are fine.
	\item[$\bullet$] No computers, phones, or any other electronic devices may be used in lecture for any reason---no exceptions. Any such devices on your person must be off (e.g., not merely on silent) and put completely away. Those who do not respect this requirement will be asked to leave class.
	\item[$\bullet$] Attendance is mandatory (and part of your participation grade). If you missed a class, please get the notes from another student. I do not offer make up sessions for students who are absent.
	\item[$\bullet$] Please, follow the `Email Etiquette' I have \href{http://www.hectorbahamonde.com/resources/}{posted} in my website.
\end{itemize}



\subsection*{Requirements and Evaluations}

\begin{enumerate}

	% Participation
	\item {\bf Readings, Participation, Attendance and Pop quizzes}: 15 \%.
	\\
	\\
	I expect you to keep up with the readings over the course of the semester. I employ an interactive lecture style, and you will need to have done the readings in order to participate. There will be a number of pop quizzes during the semester, particularly during the first part of the semester.  Quizzes will be short (3-5 minutes), completed at any point of the class, and designed to make sure everyone is keeping up with the readings and lecture. There will be no make-up quizzes. If you are absent (or late) from class that day, you will get a $0$ on that quiz. 
	\\
	\\
	When reading the class materials, you should locate the main argument, strengths, weaknesses, and other issues that are of concern. As you read through the material, think about the following questions: \emph{What is the cause and what is the effect? What makes the theory `move,' is it individuals? institutions? (ir)rationality? Does/do the author/s have a strong research design/methodology to support the paper's argument?} 
	\\
	\\
	On average, students are expected to put in approximately 10-12 hours of work per week for a four-credit class, as per U.S. Department of Education guidelines.  Since you will be spending 2.5 hours in the classroom, this means you should be working about 7.5-9.5 hours per week for this course {\bf outside} of the classroom. If you find that you are spending more than 12 hours per week on the class, please see me to discuss strategies to read more efficiently. 


	% midterm
	\item {\bf One in-class midterm exam, \underline{March 2}}: 25 \%. 
	\\
	\\
	You must take the exam at the scheduled time. There will be no make-ups, unless you have a documented medical excuse. (Documented) Medical excuses are the only type of exceptions that will be accepted.
	\\
	\\
	The exam will be a closed-book exam, covering material from the entire semester, {\bf up to February 23}. The format of the exam will be discussed on February 23, when we will review the material for the test. Please note, exam questions will be drawn \emph{both} from the readings \emph{and} lectures.


	% essay
	\item {\bf An essay of 6-8 pages in length, \underline{April 11}}: 25\%. 
	\\
	\\
	I take writing very seriously. I therefore strongly suggest that you begin your essay early, edit multiple drafts, and proofread carefully before turning it in. Grammar, diction, and style all shape the effectiveness of your writing and, as a result, will affect your grade. Consult William Strunk, Jr., and E. B. White, \emph{The Elements of Style}, for helpful hints regarding written expression. Joseph M. Williams and Gregory G. Colomb, The Craft of Argument (New York: Longman, 2003), provides an excellent overview of the art of effective persuasive writing.
	\\
	\\
	Topics will be assigned by me on {\bf March 5}. The essay is due in hard copy at the beginning of class, and no later than 8.15 am on {\bf April 11}. This assignment covers material from the entire semester, up to {\bf March 23}. Turning it in before the due date is OK, but \emph{not} afterwards. {\bf Late essays will not be accepted, and will be graded with a $0$}. There will \emph{not} be exceptions or extensions. No electronic copies of any kind will be accepted.
	\\
	\\
	I encourage you to see me in my office hours \emph{before} the due date. If you want, \href{mailto:\filetext}{send me} your draft via email, then \href{https://calendly.com/bahamonde/officehours}{make an appointment}. That way I will be able to give you feedback on your work before the due date. If you want to receive comments from me, please allow plenty of time for me to read your draft, and time to meet you. Consider also that your classmates will do the same. Consequently, plan accordingly.

	\item {\bf One cumulative in-class final exam}: 35 \%. 
	\\
	\\
	The cumulative final exam will be a closed-book exam covering material from the entire semester. The format of the exam will be discussed by the end of the semester. Exam questions will be drawn both from the readings and lectures. The final exam is set by the registrar. Hence, both place and time are TBA. There will not be exceptions. We will also schedule an in-class review session for {\bf April 27}.

\end{enumerate}




\subsection*{Grading}

This course will be grade according to the following scale: 
A: $\geq$ 93, A-: 90-92, B+: 87-89, B: 83-86, B-: 80-82, C+: 77-79, C:73-76, C-: 70-72, D+: 67-69, D:63-66, D-: 60-62, and F: $\leq$ 59. 

\subsection*{Disputing Grades}

I am happy to go over any exam or paper with you. Request for re-grading, though, must be done in writing. Please refer to my \href{https://github.com/hbahamonde/hbahamonde.github.io/raw/master/resources/ReGrade_Policy.pdf}{re-grading policy}.



\subsection*{Academic Integrity}
In accordance with Tulane University policy on Academic Integrity, you are expected to fully comply with the school's \href{https://college.tulane.edu/code-of-academic-conduct}{\texttt{policies}}. 


\subsection*{Students with Disabilities}
Students with disabilities who require accommodation should check with the \href{https://accessibility.tulane.edu/}{\texttt{Goldman Center for Student Accessibility}}.


\subsection*{Absence from Exams}


There will be no make-up exams unless you have a \emph{documented} {\bf medical} emergency. If at all possible, I need to be notified before the exam of your inability to take it. Absence from an exam because of travel plans will not be excused. Make travel plans accordingly. 


\subsection*{Office Hours}

I have an open-doors policy, feel free to stop by my office at any time. However, you might want to minimize the risks that I am not there, or can't meet you that day. I advice you then to \href{https://calendly.com/bahamonde/officehours}{\texttt{schedule time with me}} using my automatic scheduler. I think fixed office hours do not work because ... well, they are \emph{fixed}. I prefer flexibility. Hence, you can see me any day/time that's available during the week. Do not send me a reminder as I will receive an alert: If the time spot is available, I am happy to see you there.



\subsection*{Schedule}

\begin{enumerate}

\item[] {\bf Introduction}
	\begin{itemize}
		\item {\bf January 17}
			\begin{itemize}
				\item[$\bullet$] Overview of syllabus, course requirements, and introduction to the course
			\end{itemize}
	\end{itemize}


\item {\bf Comparative Politics as a Methodology/Subfield}
	\begin{itemize}
		\item {\bf January 19: `Scratching the Surface'}
		\begin{itemize}
			\item[$\bullet$] Arend Lijphart. Comparative Politics and the Comparative Method. American Political Science Review, 65(3), 682-693. \texttt{DOI:10.2307/1955513}. \url{https://www.cambridge.org/core/journals/american-political-science-review/article/comparative-politics-and-the-comparative-method/A326138E114805EF7E1B72F60EBD4295}.
			\item[$\bullet$] Gerardo Munck and Richard Snyder. Debating the Direction of Comparative Politics: An Analysis of Leading Journals. Comparative Political Studies, 40(1):5-31, 2007.
		\end{itemize}
		
		\item {\bf January 22: `An Important Debate'}
			\begin{itemize}
				\item[$\bullet$] James Mahoney. Debating the State of Comparative Politics: Views From Qualitative Research. Comparative Political Studies, 40(1):32-38, 2007. DOI: \texttt{10.1177/0010414006294816}. URL: \url{http://cps.sagepub.com/cgi/doi/10.1177/0010414006294816}.
				% better assign this one in the future. or at least, include this piece regardless.
				%\item[$\bullet$] James Mahoney. A Tale of Two Cultures: Contrasting Quantitative and Qualitative Research. Political Analysis, 14(3): 227-249.
				\item[$\bullet$] Barbara Geddes. How the Cases You Choose Affect the Answers You Get: Selection Bias in Comparative Politics. Political Analysis, 2(1):131-150, 1990.
			\end{itemize}

		\item {\bf January 24: `Two Important Concepts'}
			\begin{itemize}	
				\item[$\bullet$] Giovanni Capoccia and Daniel Kelemen. The Study of Critical Junctures Theory, Narrative, and Counterfactuals in Historical Institutionalism. World Politics, 59(April):341-369, 2007.
				% Assign this too: Clio and the Economics of QWERTY. It's 5 pages short.
				\item[$\bullet$] Gerardo Munck and Jay Verkuilen. Conceptualizing and Measuring Democracy: Evaluating Alternative Indices. Comparative Political Studies, 35(1):5-34, 2002.
			\end{itemize}



		\item {\bf January 26: `Gaining Perspective': Why and How Do We Compare?}
			\begin{itemize}	
				\item[$\bullet$] In-class video lecture. (1) {\bf Institutions and Historical Experiments}: \href{https://www.youtube.com/watch?v=jsZDlBU36n0}{`Why Nations Fails,' by James Robinson}. (2) {\bf Genes and `Natural' Experiments}: \href{https://www.youtube.com/watch?v=mVVeCOuh7FQ}{`The Genetics of Politics,' by Rose McDermott} [no readings assigned].
			\end{itemize}

	\end{itemize}


\item {\bf The State}
	\begin{itemize}
		%\item[$\bullet$] \emph{Principles}. Ch. 4.
		
	\item {\bf January 29: `The Origins: Theory and Empirics'}
		\begin{itemize} 
			\item[$\bullet$] Mancur Olson. Dictatorship, Democracy, and Development. The American Political Science Review, 87(3): 567-576, 1993. DOI: \texttt{10.2307/2938736}. \url{http://www.jstor.org/stable/2938736}.
			\item[$\bullet$] Tilly, Charles. 1985. `War Making as Organized Crime.' In \emph{Bringing the State Back In} edited by P. Evans, D. Rueschemeyer and T. Skocpol. New York: Cambridge University Press.\\
			{\color{brown}[\faGlobe: {\bf Europe}].}
		\end{itemize}
		

	\item {\bf January 31: `Beyond Europe'}
		\begin{itemize} 
			\item[$\bullet$] Miguel Angel Centeno. Blood and Debt: War and Taxation in Nineteenth-Century Latin America. American Journal of Sociology, 102(6):1565-1605, 1997.\\
			{\color{brown}[\faGlobe: {\bf Latin America}].}

			\item[$\bullet$] Francis Fukuyama. Reflections on Chinese Governance. Journal of Chinese Governance, 1(3):379-391, 2016. DOI: \texttt{10.1080/23812346.2016.1212522}. URL: \url{https://www.tandfonline.com/doi/full/10.1080/23812346.2016.1212522}\\
			{\color{brown}[\faGlobe: {\bf China}].}

		\end{itemize}

	\item {\bf  February 2: `Gaining Perspective': What Do Political Theorists Have to Say?}
		\begin{itemize} 
			\item[$\bullet$] Short lecture about Hobbes and Locke by the professor [no readings assigned].
		\end{itemize}


	\item {\bf February 5: `The \emph{Infrastructural} Power of the State'}
		\begin{itemize}
			\item[$\bullet$] Michael Mann. The Autonomous Power of the State: Its Origins, Mechanisms and Results. European Journal of Sociology, 25(2): 109-136, 1984. DOI: \texttt{10.1017/S0003975600004239}.
			\item[$\bullet$] Dan Slater. Can Leviathan be Democratic? Competitive Elections, Robust Mass Politics, and State Infrastructural Power. Studies in Comparative International Development, 43(3-4): 252-272, 2008. DOI: \texttt{10.1007/s12116-008-9026-8}. \url{http://link.springer.com/10.1007/s12116-008-9026-8}.\\
			{\color{brown}[\faGlobe: {\bf Southeast Asia}].}

		\end{itemize}
	\end{itemize}



 

\item {\bf Political Regimes: Democracy}
	\begin{itemize}
		\item {\bf February 7: Pre-Modern and Contemporaneous Comparative Exercises}
			\begin{itemize}
				\item[$\bullet$] Aristotle. \href{https://socialsciences.mcmaster.ca/econ/ugcm/3ll3/aristotle/Politics.pdf}{The Politics}. Book III (1-8).%, Book IV (1-12), Book 5 (1-4). // Include these next time.
				\item[$\bullet$] Arend Lijphart. \href{https://e-edu.nbu.bg/pluginfile.php/830138/mod_resource/content/1/Lijphart%2C%20A.%20Patterns%20of%20Democracy%20-%20Government%20Forms%20and%20Performance%20in%20Thirty-Six%20Countries%20%282012%29.pdf}{Patterns of Democracy: Government Forms and Performance in Thirty-Six Countries}. Yale University Press, 2nd edition, 2012. Chapters 1-3. % next time: organize a debate between parliamentary v. presidential regimes (perhaps add more bibliography.)
			\end{itemize}
	\end{itemize}


\vspace{3mm}{\bf {\color{blue}February 9 to February 14: Mardi Gras Break. Have fun, see you February 16.}}\vspace{3mm}

\begin{itemize}

		\item {\bf February 16: Hybrid Regimes (1)}
			\begin{itemize}
				\item[$\bullet$] Steven Levitsky and Lucan Way. \href{http://muse.jhu.edu/article/17196}{The Rise of Competitive Authoritarianism}. Journal of Democracy, 13(2):51-65, 2002. DOI: \texttt{10.1353/jod.2002.0026}.
				\item[$\bullet$] `Gaining Perspective' (1): In-class movie about Mexico's PRI. `Herod's Law' (Mexico, 1999)\\
			{\color{brown}[\faGlobe: {\bf Mexico}].}

			\end{itemize}

		\item {\bf February 19: Hybrid Regimes (2)}
			\begin{itemize}
				\item[$\bullet$] `Gaining Perspective' (2): In-class movie about Mexico's PRI. `Herod's Law' (Mexico, 1999) [no readings assigned].\\
			{\color{brown}[\faGlobe: {\bf Mexico}].}

			\end{itemize}



		\item {\bf February 21: Concepts, A Methodological Addendum}
			\begin{itemize}
				\item[$\bullet$] Giovanni Sartori. Concept Misformation in Comparative Politics. The American Political Science Review, 64(4):1033-1053, 1970.
				\item[$\bullet$] David Collier and Steven Levitsky. Democracy with Adjectives: Conceptual Innovation in Comparative Politics. World Politics, 49(April):430-451, 1997.
			\end{itemize}
		
	
	\end{itemize}

\vspace{3mm}{\bf {\color{blue}Midterm Review: February 23 (No readings, bring your questions).}}\vspace{3mm}

	\begin{center}$\uparrow$ {\color{blue}For the test, keep calm, and study everything that's above} $\uparrow$\end{center}
	\begin{center}\Smiley{{\color{blue}You can do this}}\Smiley{}\end{center}


	\begin{itemize} 
		\item[] {\bf February 26: Why do we care?}
			\begin{itemize} 
				\item[$\bullet$] Daron Acemoglu and James Robinson. \emph{Why Nations Fail: The Origins of Power, Prosperity, and Poverty}, 2012. Ch. 3.
			\end{itemize}
	\end{itemize}

	\begin{itemize} 
		\item {\bf February 28: Does Democracy Make a Difference?}
			\begin{itemize} 
				%\item[$\bullet$] Principles, Ch. 9, Democracy or Dictatorship: Does It Make a Difference? 
				%\item[$\bullet$] Principles, Ch. 11, Problems with Group Decision Making.
				%\item[$\bullet$] Daron Acemoglu, Suresh Naidu, Pascual Restrepo, and James Robinson. Democracy Does Cause Growth. Technical report, National Bureau of Economic Research, Cambridge, MA, URL: \url{http://www.nber.org/papers/w20004.pdf}.
				\item[$\bullet$] Adam Przeworski and Fernando Limongi. Political Regimes and Economic Growth. Journal of Economic Perspectives, 7(3): 51-69, 1993. DOI: \texttt{10.1257/jep.7.3.51}. \url{http://pubs.aeaweb.org/doi/abs/10.1257/jep.7.3.51}.
				\item[$\bullet$] Michael Ross. Is Democracy Good for the Poor? American Journal of Political Science, 50(4):860-874,  2006. DOI: \texttt{10.1111/j.1540-5907.2006.00220.x}. URL: \url{http://doi.wiley.com/10.1111/j.1540-5907.2006.00220.x}.
			\end{itemize}
	\end{itemize}


\vspace{2mm}{\bf {\color{blue}March 2: Midterm, same classroom, same time.}}\vspace{2mm}\\
\vspace{2mm}{\bf {\color{blue}March 5: Distribute Essay Topics and Instructions in Class.}}\vspace{2mm}


\item {\bf Democracy: Different Determinants and Explanations}
	\begin{itemize} {\bf March 5: Modernization Theory}
	%	\item
				\begin{itemize}
					\item[$\bullet$] Seymour Martin Lipset. Some Social Requisites of Democracy: Economic Development and Political Legitimacy. The American Political Science Review, 53(1):69-105, 1959.
					\item[$\bullet$] Adam Przeworski and Fernando Limongi. Modernization: Theories and Facts. World Politics, 49(2):155-183, 1997.
				\end{itemize}

		\item {\bf March 7: Social Origins of Democracy}
				\begin{itemize}
					\item[$\bullet$] Barrington Moore, Jr., \emph{Social Origins of Dictatorship and Democracy}, Beacon, 1964. Preface, Chs. 1 and 7.\\
			{\color{brown}[\faGlobe: {\bf England}].}

				\end{itemize}

		\item {\bf March 9: `Gaining Perspective': ``Why the Industrial Revolution Happened Here''}:
				\begin{itemize}
					\item[$\bullet$] \href{https://www.youtube.com/watch?v=UM2Aw4kmA0s}{Documentary} in class [no readings assigned].
				\end{itemize}
				
		\item {\bf March 12: Economic Origins of Democracy}
 				\begin{itemize}
					\item[$\bullet$] Daron Acemoglu and James Robinson. Economic Origins of Dictatorship and Democracy, Cambridge, 2013. Chp. 1 and 2.
					\item[$\bullet$] `Gaining Perspective': Movie `NO' (1). In-class movie about the Chilean transition to democracy.\\
			{\color{brown}[\faGlobe: {\bf Chile}].}
				\end{itemize}

		\item {\bf March 14: `An Important Debate'}
 				\begin{itemize}
					\item[$\bullet$] Carles Boix. Democracy and Redistribution, Cambridge, 2003. Introduction and Ch. 3 ('Historical Evidence').
					\item[$\bullet$] Stephan Haggard and Robert Kaufman. Inequality and Regime Change: Democratic Transitions and the Stability of Democratic Rule. American Political Science Review, 106(3): 495-516, 2012.
				\end{itemize}

		\item {\bf March 16: `Gaining Perspective': Movie `NO' (2)}
 				\begin{itemize}
					\item[$\bullet$] In-class movie about the Chilean transition to democracy [no readings assigned].\\
			{\color{brown}[\faGlobe: {\bf Chile}].}

				\end{itemize}

	\end{itemize}




		%\item[$\bullet$] Principles, Ch. 6, The Economic Determinants of Democracy and Dictatorship
		%\item[$\bullet$] Principles, Ch. 7, Cultural Determinants of Democracy and Dictatorship
		%\item[$\bullet$] Daron Acemoglu and James Robinson. Why Did The West Extend The Franchise? Democracy, Inequality, and Growth in Historical Perspective. The Quarterly Journal of Economics, 115(4):1167-1199, 2000. DOI: \texttt{10.1162/003355300555042}.
		%\item[$\bullet$] Guillermo O'Donnell. Modernization and Bureaucratic-Authoritarianism: Studies in South American Politics. Univ of California Intl, 1973. Pages TBA.
		%\item[$\bullet$] Joseph Wright, Erica Frantz, Barbara Geddes. Oil and Autocratic Regime Survival. (2013):1-20, sep 2013. \url{http://www.journals.cambridge.org/abstract S0007123413000252}. British Journal of Political Science, DOI: \texttt{10.1017/S0007123413000252}.
	%\end{itemize}



\item {\bf Political Regimes: Authoritarianism}
\begin{itemize}	
	\item {\bf March 19: Surveying Authoritarian Regimes}
			\begin{itemize}
				\item[$\bullet$] Barbara Geddes. \href{https://www.annualreviews.org/doi/pdf/10.1146/annurev.polisci.2.1.115}{What Do We Know About Democratization After Twenty Years?} Annual Review of Political Science, 2:115-144, 1999
				

				\item[$\bullet$] Stephen Haber. \href{http://www.oxfordhandbooks.com/oxford/downloaddoclightbox/$002f10.1093$002foxfordhb$002f9780199548477.001.0001$002foxfordhb-9780199548477-e-038/Authoritarian$0020Government;jsessionid=2E89E3A5EE542CF1CE7583F64EFD4C17?nojs=true}{Authoritarian Government}. In \emph{The Oxford Handbook of Political Economy}, Oxford, 2009. 
			\end{itemize}


	\item {\bf March 21: Authoritarian Survival}
			\begin{itemize}
				\item[$\bullet$] Bruce Bueno de Mesquita and Alastair Smith. Dictator's Handbook, Why Bad Behavior Is Almost Always Good Politics. Public Affairs, 2012. Ch. 3 "Staying in Power." {\bf Also}, learn about the "\href{https://www.youtube.com/watch?v=DON-aM2tze4}{selectorate theory}." 
					% consider podcast on selectorate theory (1hr) : http://www.econtalk.org/archives/2007/02/bruce_bueno_de.html
					% consider "Top 10 Brutal Dictators You've Never Heard Of": https://www.youtube.com/watch?v=rtlssoJbCYQ
				\item[$\bullet$] Beatriz Magaloni. \href{http://cps.sagepub.com/cgi/content/abstract/41/4-5/715}{Credible Power-Sharing and the Longevity of Authoritarian Rule}. Comparative Political Studies, 41(4-5):715-741, 2008. DOI: \texttt{10.1177/0010414007313124}.
			\end{itemize}
\end{itemize}



\begin{itemize}
	\item {\bf March 23: `Gaining Perspective': Documentary `How To Stage a Coup'}
			\begin{itemize}
				\item[$\bullet$] In-class documentary about the most iconic coups in history [no readings assigned].
			\end{itemize}
\end{itemize}


\vspace{2mm}{\bf {\color{blue}Spring Break. Have fun. In the meantime, professor will be presenting at the \emph{Western Political Science Association Conference} in San Francisco.}}\vspace{2mm}


\vspace{1mm}{\bf {\color{blue}Professor will be at the \emph{Midwest Political Science Association Conference} in Chicago. See you April 11.}}\vspace{1mm}


		%\item[$\bullet$] Carles Boix and Susan Stokes. Endogenous Democratization. World Politics, 55 (4):517-549, 2003.
		%\item[$\bullet$] Ben Ansell and David Samuels. Inequality and Democratization: A Contractarian Approach. Comparative Political Studies, 43(12):1543-1574, 2010. DOI: \texttt{10.1177/0010414010376915}. URL: \url{http://cps.sagepub.com/cgi/doi/10.1177/0010414010376915}.
		%\item[$\bullet$] Milan Svolik. Authoritarian Reversals and Democratic Consolidation. American Political Science Review, 102(2):153-168, 2008. DOI: \texttt{10.1017/S0003055408080143}. URL: \url{http://www.journals.cambridge.org/abstract S0003055408080143}.

\vspace{2mm}{\bf {\color{blue}April 11: Paper Due at 8.15 AM (Central Time) in Hard Copy.}}\vspace{2mm}


\item {\bf Parties and Elections}
	\begin{itemize} 
		\item {\bf April 11: Origins and Consequences of Electoral Systems}
			\begin{itemize} 
				\item[$\bullet$] Carles Boix. Setting the Rules of the Game: The Choice of Electoral Systems in Advanced Democracies. The American Political Science Review, 93(3):609-624, 1999. ISSN 1556-5068. DOI: \texttt{10.2139/ssrn.159213}. URL: \url{http://www.ssrn.com/abstract=159213}.
				\item[$\bullet$] Arend Lijphart. The Political Consequences of Electoral Laws, 1945-85. The American Political Science Review, 84(2):481-496, 1990.
			\end{itemize}
		%\item[$\bullet$] Principles, Ch. 13, Elections and Electoral Systems. 
		%\item[$\bullet$] Principles, Ch. 14, Social Cleavages and Party Systems.
		%\item[$\bullet$] Seymour Martin Lipset and Stein Rokkan. Party Systems and Voter Alignments: Cross-National Perspectives. Free Press, 1967. Pages TBA.
		\item {\bf April 13: `Gaining Perspective': U.S. Elections In Historical Perspective}
			\begin{itemize}
				\item[$\bullet$] TBA. Tentative: In-class video lecture ``\href{https://www.youtube.com/watch?v=gF8CJSQf238}{Democracy's Failure to Perform},'' by Francis Fukuyama (28:51 onwards).\\
			{\color{brown}[\faGlobe: {\bf United States}].}
			\end{itemize}
	\end{itemize}


\item {\bf Parties and Elections: Selected Topics}
	\begin{itemize} 
		\item {\bf April 16: Clientelism and Vote-Buying}
		\begin{itemize}
			\item[$\bullet$] Herbert Kitschelt. Linkages between Citizens and Politicians in Democratic Polities. Comparative Political Studies, 33(6-7):845-879, sep 2000. ISSN 0010-4140. DOI: \texttt{10.1177/001041400003300607}. URL: \url{http://cps.sagepub.com/cgi/doi/10.1177/001041400003300607}.
			\item[$\bullet$] Javier Auyero. The Logic of Clientelism in Argentina: An Ethnographic Account. Latin American Research Review, 35(3):55-81, 2000. URL: \url{http://www.jstor.org/stable/2692042}.\\
			{\color{brown}[\faGlobe: {\bf Argentina}].}

		\end{itemize}
	\end{itemize}


	\begin{itemize} 
		\item {\bf April 18: Comparative Political Behavior}
		\begin{itemize}
			\item[$\bullet$] Richard Lau, Parina Patel, Dalia Fahmy, and Robert Kaufman. Correct Voting Across Thirty-Three Democracies: A Preliminary Analysis. British Journal of Political Science, 44(02):239-259, 2013. DOI: \texttt{10.1017/S0007123412000610}. URL: \url{http://www.journals.cambridge.org/abstract S0007123412000610}.
			\item[$\bullet$] Ryan Carlin, Gregory Love and Elizabeth Zechmeister. \href{ http://www.jstor.org/stable/43664117}{Trust Shaken: Earthquake Damage, State Capacity, and Interpersonal Trust in Comparative Perspective}. Comparative Politics, 46(4), 419-437, 2014.
		\end{itemize}
	\end{itemize}


	\begin{itemize} 
		\item {\bf April 20: `Gaining Perspective': Street Politics, Bribery, and Political Violence in New Jersey}
		\begin{itemize}
			\item[$\bullet$] In-class documentary `Street Fight' about the 2002 mayoral race in Newark, New Jersey [no readings assigned].\\
			{\color{brown}[\faGlobe: {\bf United States}].}
		\end{itemize}
	\end{itemize}




\item {\bf Development and Colonialism}
	\begin{itemize}
		\item {\bf April 23: Competing Causes of Development (1)}
		\begin{itemize}
			\item[$\bullet$] John Gallup, Jeffrey Sachs, and Andrew Mellinger. Geography and Economic Development. Technical report, National Bureau of Economic Research, Cambridge, MA, 1998. URL: \url{http://www.nber.org/papers/w6849.pdf}.

			\item[$\bullet$] Daron Acemoglu, Simon Johnson, and James Robinson. Reversal Fortune: Geography and Institutions in the Making of the Modern World Income Distribution. The Quarterly Journal of Economics, 117(4):1231-1294, 2002.
		\end{itemize}

		\item {\bf April 25: Competing Causes of Development (2)}
			\begin{itemize}
				\item[$\bullet$] Kenneth Sokoloff and Stanley Engerman. History Lessons: Institutions, Factor Endowments, and Paths of Development in the New World. Journal of Economic Perspectives, 14(3):217-232, 2000. DOI: \texttt{10.1257/jep.14.3.217}. URL: \url{http://pubs.aeaweb.org/doi/abs/10.1257/jep.14.3.217}.
		%\item[$\bullet$] Daron Acemoglu, Simon Johnson, and James Robinson. The Colonial Origins of Comparative Development: An Empirical Investigation. American Economic Review, 91(5):1369-1401, 2001. DOI: \texttt{10.1257/aer.91.5.1369}. URL: \url{http://pubs.aeaweb.org/doi/abs/10.1257/aer.91.5.1369}.
				\item[$\bullet$] Matthew Lange, James Mahoney, and Matthias vom Hau. Colonialism and Development: A Comparative Analysis of Spanish and British Colonies. American Journal of Sociology, 111(5):1412-1462, 2006. DOI: \texttt{10.1086/499510}. URL: \url{http://www.journals.uchicago.edu/doi/10.1086/499510}.
		\end{itemize}
	\end{itemize}
  


\vspace{2mm}{\bf {\color{blue}April 27: Review for Final Exam. Bring your questions. No readings assigned.}}\vspace{2mm}

	\begin{center}$\uparrow$ {\color{blue}For the final exam, keep extra calm, and study everything that's above} $\uparrow$\end{center}
	\begin{center}\Smiley{{\color{blue}You can do this}}\Smiley{}\end{center}

\vspace{2mm}{\bf {\color{blue}Final Exam (Date TBA).}}\vspace{2mm}

%\item {\bf Political Violence}
%	\begin{itemize}
%		\item[$\bullet$] Fearon, James D., and David D. Laitin. 2003. Ethnicity, Insurgency, and Civil War. American Political Science Review 97 (1):75-90.
%		\item[$\bullet$] Wilkinson, Steven I. 2004. Votes and Violence: Electoral Competition and Ethnic Riots in India. Cambridge: Cambridge University Press, Chapters 1 and 2.
%	\end{itemize}

%\item {\bf Ideas v. Structure v. Psychology}
%	\begin{itemize}
%		\item[$\bullet$] Ana De La O and Jonathan Rodden. Does Religion Distract the Poor?: Income and Issue Voting Around the World. Comparative Political Studies, 41(4-5):437-476, 2008. DOI: \texttt{10.1177/0010414007313114}. URL: \url{http://cps.sagepub.com/cgi/doi/10.1177/0010414007313114}.
%		\item[$\bullet$] David Laitin. National Revivals and Violence. European Journal of Sociology, 36(01):3, 1995. DOI: \texttt{10.1017/S0003975600007098}. URL: \url{http://www.journals.cambridge.org/abstract S0003975600007098}.
%		\item[$\bullet$] David Stasavage. Representation and Consent: Why They Arose in Europe and Not Elsewhere. Annual Review of Political Science, 19(1):145-162, 2016. DOI: \texttt{10.1146/annurev-polisci-043014-105648}. URL: \url{http://www.annualreviews.org/doi/abs/10.1146/annurev-polisci-043014-105648 http://www.annualreviews.org/doi/10.1146/annurev-polisci-043014-105648}.
%		\item[$\bullet$] Gian Vittorio, Claudio Barbaranelli, and Philip Zimbardo. Profiles Personality and Political Parties. Political Psychology, 20(1):175-197, 1999.

%	\end{itemize}


\end{enumerate}


 









%\bibliographystyle{plainnat}
%\bibliography{/Users/hectorbahamonde/RU/Bibliografia_PoliSci/Bahamonde_BibTex2013}

\end{document}