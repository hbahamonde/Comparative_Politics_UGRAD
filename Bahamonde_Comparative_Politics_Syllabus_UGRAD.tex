% LaTeX Curriculum Vitae Template
%
% Copyright (C) 2004-2009 Jason Blevins <jrblevin@sdf.lonestar.org>
% http://jblevins.org/projects/cv-template/
%
% You may use use this document as a template to create your own CV
% and you may redistribute the source code freely. No attribution is
% required in any resulting documents. I do ask that you please leave
% this notice and the above URL in the source code if you choose to
% redistribute this file.

\documentclass[letterpaper]{article}

\usepackage{hyperref}
\usepackage{geometry}

% Comment the following lines to use the default Computer Modern font
% instead of the Palatino font provided by the mathpazo package.
% Remove the 'osf' bit if you don't like the old style figures.
\usepackage[T1]{fontenc}
\usepackage[sc,osf]{mathpazo}

% Set your name here
\def\name{Introduction to Comparative Politics}

% Replace this with a link to your CV if you like, or set it empty
% (as in \def\footerlink{}) to remove the link in the footer:
\def\footerlink{}
% \href{http://www.hectorbahamonde.com}{www.HectorBahamonde.com}

% The following metadata will show up in the PDF properties
\hypersetup{
  colorlinks = true,
  urlcolor = blue,
  pdfauthor = {\name},
  pdfkeywords = {political science, comparative politics},
  pdftitle = {\name: Syllabus},
  pdfsubject = {Syllabus},
  pdfpagemode = UseNone
}

\geometry{
  body={6.5in, 8.5in},
  left=1.0in,
  top=1.25in
}

% Customize page headers
\pagestyle{myheadings}
\markright{{\tiny \name}}
\thispagestyle{empty}

% Custom section fonts
\usepackage{sectsty}
\sectionfont{\rmfamily\mdseries\Large}
\subsectionfont{\rmfamily\mdseries\itshape\large}

% Don't indent paragraphs.
\setlength\parindent{0em}

% Make lists without bullets
\renewenvironment{itemize}{
  \begin{list}{}{
    \setlength{\leftmargin}{1.5em}
  }
}{
  \end{list}
}

\begin{document}

% Place name at left
%{\huge \name}

% Alternatively, print name centered and bold:
\centerline{\huge \bf \name}

\vspace{0.25in}

\begin{minipage}{0.45\linewidth}
  Rutgers University, New Brunswick \\
  Political Science Department \\
  Hickman Hall \\
  New Brunswick, NJ 08901\\
  \\
  \\

\end{minipage}
\hspace{4cm}\begin{minipage}{0.45\linewidth}
  \begin{tabular}{ll}
{\bf Last updated}: \today. \\
 {\bf Download last version} \href{https://github.com/hbahamonde/Comparative_Politics_UGRAD/raw/master/Bahamonde_Comparative_Politics_Syllabus_UGRAD.pdf}{here}.\\
   {\bf {\color{red}{\scriptsize Not intended as a definitive version}}} %\\
    \\
    \\
    \\
    \\
    \\
  \end{tabular}
\end{minipage}

\vspace{-5mm}
{\bf Instructor}: H\'ector Bahamonde\\
\texttt{e:}\href{mailto:hector.bahamonde@rutgers.edu}{\texttt{hector.bahamonde@rutgers.edu}}\\
\texttt{w:}\href{http://www.hectorbahamonde.com}{\texttt{www.hectorbahamonde.com}}\\
{\bf Location}: Classroom.\\
{\bf Office Hours}: Make an appointment \href{https://calendly.com/bahamonde/officehours}{\texttt{here}}.\\
{\bf Class Website and Materials}: click \href{https://github.com/hbahamonde/Comparative_Politics_UGRAD}{\texttt{here}}.

\subsection*{Overview and Objectives}

\emph{What does the state have so special that makes it so effective to organize us politically? Why are some societies more violent than others? What can we learn by comparing different electoral systems? Is religion (or another form of `culture') responsible for explaining democratic failures? Do diverse societies `do better' than cohesive societies? What can we learn by `comparing' countries, elections, events, economies or political leaders?} These and other questions are still subject of great debate in comparative politics. This {\bf {\color{blue}undergraduate-level course}} is intended as an introduction to comparative politics. The papers and chapters will draw from what call `the core' that defines our subfield. Comparative politics is both a \emph{substantive} as well as a \emph{methodological} area of research. That is, we are not only interested in \emph{what} is happening/has happened, but also in \emph{how} we learn and define those things. You will quickly realize that `concepts' are fundamental. For example, we are still debating what a `democracy' is since we don't agree on what are the constitutive elements that define what a `democracy' is. Well, we will spend some time talking about some cases and also discussing some important methodological issues. You will quickly realize that comparative politics is quite a \emph{flexible} subfield. Any country is of interest for us. Single-cases as well as regional approaches (i.e. `Africa,' `Latin America,' etc.) are acceptable. A number of methodologies and approaches are possible. Any time period and (almost any) topic are interesting for us: from the rise of Babylonian state to the exit of the United Kingdom from the European Union. And such, we comparativists borrow from sociology, economics, history, political theory, among others. 


I hope this course catches your attention, in the hope you continue taking more comparative politics courses. Most of all, I hope you see what a diverse world, practices (informal and formal) we have. 

\subsection*{Course Learning Objectives}
 
Upon successful completion of this course, you will be able to:

\begin{itemize}
	\item[$\bullet$] Acquire an understanding of the main comparative politics theories and topics.
	\item[$\bullet$] Use the comparative method and analysis in the political science literature.
	\item[$\bullet$] Consume `critically' the comparative politics literature.
\end{itemize}



\subsection*{Requirements}

In this course we will cover the key concepts and theoretical debates in a very large sub-field in political science. Students will be expected to complete the required readings each week, attend the lectures, participate in class discussions and take careful notes. When readings the class materials, you should locate the main argument, strengths, weaknesses, and other issues that are of concern. As you read through the material, think about the following questions: \emph{What is the cause and what is the effect? What makes the theory `move,' is it individuals? institutions? (ir)rationality? Does/do the author/s have a strong research design/methodology to support the paper's argument?}


\subsection*{Evaluation}


\begin{itemize}
	\item[$\bullet$] {\bf Two midterm papers}: 25 \%.
	\item[$\bullet$] {\bf Final exam paper}: 25 \%.
	\item[$\bullet$] {\bf Participation}: 25 \%.
\end{itemize}


\subsection*{Academic Integrity}
In accordance with Rutgers University policy on Academic Integrity, you are expected to fully comply with the school's policies.  Please see this \href{http://academicintegrity.rutgers.edu}{\texttt{link}}.


\subsection*{Students with Disabilities}
Students with disabilities who require accommodation should review the following statement from the Office of Disability Services \href{https://ods.rutgers.edu/faculty/syllabus}{\texttt{link}}.


\subsection*{Absence from Exams}

Only a note from your college dean stipulating a medical or family emergency will be acceptable as an excuse for missing an exam. If at all possible, I need to be notified before the exam of your inability to take it. Absence from an exam because of travel plans will not be excused.


\subsection*{Office Hours}

I have an open-doors policy, feel free to stop by my office at any time. However, you might want to minimize the risks that I am not there. I advice you then to schedule time with me using my automatic scheduler. I think fixed office hours do not work because ... well, they are fixed. I prefer flexibility. Hence, you can see me any day/time that's available during the week. Do not send me a reminder as I will receive an alert: If the time spot is available, I am happy to see you there. Please follow this \href{https://calendly.com/bahamonde/officehours}{\texttt{link}}.


\subsection*{Cell Phones} 

Make sure your cell phones are turned OFF before entering class.


%\section*{Readings}
%
%The following are required books.
%
%\subsection*{Required Books}
%
%	\begin{itemize}
%		\item[$\bullet$] William Clark, Matt Golder, and Sona Golder. 2013. \emph{Principles of Comparative Politics}. 2nd ed. Thousand Oaks, CA 91320 CQ Press.
%	\end{itemize}


\subsection*{Schedule}

\begin{enumerate}

\item {\bf The State}
	\begin{itemize}
		%\item[$\bullet$] \emph{Principles}. Ch. 4.
		\item[$\bullet$] Tilly, Charles. 1985. `War Making as Organized Crime.' In \emph{Bringing the State Back In} edited by P. Evans, D. Rueschemeyer and T. Skocpol. New York: Cambridge University Press.
		\item[$\bullet$] Miguel Angel Centeno. Blood and Debt: War and Taxation in Nineteenth-Century Latin America. American Journal of Sociology, 102(6):1565-1605, 1997.
		\item[$\bullet$] Francis Fukuyama. Reflections on Chinese Governance. Journal of Chinese Governance, 1(3):379-391, 2016. doi: \texttt{10.1080/23812346.2016.1212522}. URL: \url{https://www.tandfonline.com/doi/full/10.1080/23812346.2016.1212522}

	\end{itemize}



\item {\bf Political Violence}
	\begin{itemize}
		\item[$\bullet$] Fearon, James D., and David D. Laitin. 2003. Ethnicity, Insurgency, and Civil War. American Political Science Review 97 (1):75-90.
		\item[$\bullet$] Wilkinson, Steven I. 2004. Votes and Violence: Electoral Competition and Ethnic Riots in India. Cambridge: Cambridge University Press, Chapters 1 and 2.
	\end{itemize}


\item {\bf Political Regimes: A typology}
	\begin{itemize}
		%\item[$\bullet$] Principles, Ch. 5, Democracy and Dictatorship: Conceptualization and Measurement.
		%\item[$\bullet$] Principles, Ch 10, Varieties of Dictatorship, pp. 349-384.
		\item[$\bullet$] Giovanni Sartori. Concept Misformation in Comparative Politics. The American Political Science Review, 64(4):1033-1053, 1970.
		\item[$\bullet$] David Collier and Steven Levitsky. Democracy with Adjectives: Conceptual Innovation in Comparative Politics. World Politics, 49(April):430-451, 1997.
		\item[$\bullet$] Steven Levitsky and Lucan Way. The Rise of Competitive Authoritarianism. Journal of Democracy, 13(2):51-65, 2002. doi: \texttt{10.1353/jod.2002.0026}.
		\item[$\bullet$] Arend Lijphart. Patterns of Democracy: Government Forms and Performance in Thirty-Six Countries. Yale University Press, 2nd edition, 1999. Pages TBA.
	\end{itemize}



\item {\bf Political Regimes: Why do we care?}
	\begin{itemize}
		%\item[$\bullet$] Principles, Ch. 9, Democracy or Dictatorship: Does It Make a Difference? 
		%\item[$\bullet$] Principles, Ch. 11, Problems with Group Decision Making.
		\item[$\bullet$] Daron Acemoglu, Suresh Naidu, Pascual Restrepo, and James Robinson. Democracy Does Cause Growth. Technical report, National Bureau of Economic Research, Cambridge, MA, URL: \url{http://www.nber.org/papers/w20004.pdf}.
		\item[$\bullet$] Michael Ross. Is Democracy Good for the Poor? American Journal of Political Science, 50(4):860-874,  2006. doi: \texttt{10.1111/j.1540-5907.2006.00220.x}. URL: \url{http://doi.wiley.com/10.1111/j.1540-5907.2006.00220.x}.
	\end{itemize}


\item {\bf Political Regimes: Different Determinants and Explanations}
	\begin{itemize}
		%\item[$\bullet$] Principles, Ch. 6, The Economic Determinants of Democracy and Dictatorship
		%\item[$\bullet$] Principles, Ch. 7, Cultural Determinants of Democracy and Dictatorship
		\item[$\bullet$] Seymour Martin Lipset. Some Social Requisites of Democracy: Economic Development and Political Legitimacy. The American Political Science Review, 53(1):69-105, 1959.
		\item[$\bullet$] Daron Acemoglu and James Robinson. Why Did The West Extend The Franchise? Democracy, Inequality, and Growth in Historical Perspective. The Quarterly Journal of Economics, 115(4):1167-1199, 2000. doi: \texttt{10.1162/003355300555042}.
		\item[$\bullet$] Adam Przeworski and Fernando Limongi. Modernization: Theories and Facts. World Politics, 49(2):155-183, 1997.
		\item[$\bullet$] Guillermo O'Donnell. Modernization and Bureaucratic-Authoritarianism: Studies in South American Politics. Univ of California Intl, 1973. Pages TBA.
		\item[$\bullet$] Joseph Wright, Erica Frantz, Barbara Geddes. Oil and Autocratic Regime Survival. (2013):1-20, sep 2013. \url{http://www.journals.cambridge.org/abstract S0007123413000252}. British Journal of Political Science, doi: \texttt{10.1017/S0007123413000252}.
	\end{itemize}



\item {\bf Parties and Elections}
	\begin{itemize}
		%\item[$\bullet$] Principles, Ch. 13, Elections and Electoral Systems. 
		%\item[$\bullet$] Principles, Ch. 14, Social Cleavages and Party Systems.
		\item[$\bullet$] Seymour Martin Lipset and Stein Rokkan. Party Systems and Voter Alignments: Cross-National Perspectives. Free Press, 1967. Pages TBA.
		\item[$\bullet$] Carles Boix. Setting the Rules of the Game: The Choice of Electoral Systems in Advanced Democracies. The American Political Science Review, 93(3):609-624, 1999. ISSN 1556-5068. doi: \texttt{10.2139/ssrn.159213}. URL: \url{http://www.ssrn.com/abstract=159213}.
		\item[$\bullet$] Arend Lijphart. The Political Consequences of Electoral Laws, 1945-85. The American Political Science Review, 84(2):481-496, 1990.
	\end{itemize}


\item {\bf Parties and Elections: Some issues}
	\begin{itemize}
		\item[$\bullet$] Herbert Kitschelt. Linkages between Citizens and Politicians in Democratic Polities. Comparative Political Studies, 33(6-7):845-879, sep 2000. ISSN 0010-4140. doi: \texttt{10.1177/001041400003300607}. URL: \url{http://cps.sagepub.com/cgi/doi/10.1177/001041400003300607}.
		\item[$\bullet$] Javier Auyero. The Logic of Clientelism in Argentina: An Ethnographic Account. Latin American Research Review, 35(3):55-81, 2000. URL: \url{http://www.jstor.org/stable/2692042}.
		\item[$\bullet$] Ezequiel Gonzalez-Ocantos, Chad Kiewiet de Jonge, and David Nickerson. The Conditionality of Vote-Buying Norms: Experimental Evidence from Latin America. American Journal of Political Science, 58(1):197-211,  2014. doi: \texttt{10.1111/ajps.12047}. URL: \url{http://doi.wiley.com/10.1111/ajps.12047}.
		\item[$\bullet$] Brian Min and Miriam Golden. Electoral Cycles in Electricity Losses in India. Energy Policy, 65:619-625, 2014. doi: \texttt{10.1016/j.enpol.2013.09.060}. URL: \url{http://linkinghub.elsevier.com/retrieve/pii/S0301421513009841}.
		\item[$\bullet$] Richard Lau, Parina Patel, Dalia Fahmy, and Robert Kaufman. Correct Voting Across Thirty-Three Democracies: A Preliminary Analysis. British Journal of Political Science, 44(02):239-259, 2013. doi: \texttt{10.1017/S0007123412000610}. URL: \url{http://www.journals.cambridge.org/abstract S0007123412000610}.
	\end{itemize}




\item {\bf Democracy and Autocracy: Transition and Survival}
	\begin{itemize}
		\item[$\bullet$] Daron Acemoglu and James Robinson. A Theory of Political Transitions. American Economic Review, 91(4):938-963, 2001. doi: \texttt{10.1257/Aer.91.4.938}.
		\item[$\bullet$] Carles Boix and Susan Stokes. Endogenous Democratization. World Politics, 55 (4):517-549, 2003.
		\item[$\bullet$] Ben Ansell and David Samuels. Inequality and Democratization: A Contractarian Approach. Comparative Political Studies, 43(12):1543-1574, 2010. doi: \texttt{10.1177/0010414010376915}. URL: \url{http://cps.sagepub.com/cgi/doi/10.1177/0010414010376915}.
		\item[$\bullet$] Stephan Haggard and Robert Kaufman. Inequality and Regime Change: Democratic Transitions and the Stability of Democratic Rule. American Political Science Review, 106(03):495-516, 2012. doi: \texttt{10.1017/S0003055412000287}. URL: \url{http://www.journals.cambridge.org/abstract S0003055412000287}.
		\item[$\bullet$] Milan Svolik. Authoritarian Reversals and Democratic Consolidation. American Political Science Review, 102(2):153-168, 2008. doi: \texttt{10.1017/S0003055408080143}. URL: \url{http://www.journals.cambridge.org/abstract S0003055408080143}.
		\item[$\bullet$] Beatriz Magaloni. Credible Power-Sharing and the Longevity of Authoritarian Rule. Comparative Political Studies, 41(4-5):715-741, 2008. doi: \texttt{10.1177/0010414007313124}. URL: \url{http://cps.sagepub.com/cgi/content/abstract/41/4-5/715}.
	\end{itemize}



\item {\bf Development and Colonialism}
	\begin{itemize}
		\item[$\bullet$] John Gallup, Jeffrey Sachs, and Andrew Mellinger. Geography and Economic Development. Technical report, National Bureau of Economic Research, Cambridge, MA, 1998. URL: \url{http://www.nber.org/papers/w6849.pdf}.

		\item[$\bullet$] Daron Acemoglu, Simon Johnson, and James Robinson. Reversal Fortune: Geography and Institutions in the Making of the Modern World Income Distribution. The Quarterly Journal of Economics, 117(4):1231-1294, 2002.
		\item[$\bullet$] Kenneth Sokoloff and Stanley Engerman. History Lessons: Institutions, Factor Endowments, and Paths of Development in the New World. Journal of Economic Perspectives, 14(3):217-232, 2000. doi: \texttt{10.1257/jep.14.3.217}. URL: \url{http://pubs.aeaweb.org/doi/abs/10.1257/jep.14.3.217}.
		\item[$\bullet$] Daron Acemoglu, Simon Johnson, and James Robinson. The Colonial Origins of Comparative Development: An Empirical Investigation. American Economic Review, 91(5):1369-1401, 2001. doi: \texttt{10.1257/aer.91.5.1369}. URL: \url{http://pubs.aeaweb.org/doi/abs/10.1257/aer.91.5.1369}.
		\item[$\bullet$] Matthew Lange, James Mahoney, and Matthias vom Hau. Colonialism and Development: A Comparative Analysis of Spanish and British Colonies. American Journal of Sociology, 111(5):1412-1462, 2006. doi: \texttt{10.1086/499510}. URL: \url{http://www.journals.uchicago.edu/doi/10.1086/499510}.
	\end{itemize}
  


\item {\bf Ideas v. Structure v. Psychology}
	\begin{itemize}
		\item[$\bullet$] Ana De La O and Jonathan Rodden. Does Religion Distract the Poor?: Income and Issue Voting Around the World. Comparative Political Studies, 41(4-5):437-476, 2008. doi: \texttt{10.1177/0010414007313114}. URL: \url{http://cps.sagepub.com/cgi/doi/10.1177/0010414007313114}.
		\item[$\bullet$] David Laitin. National Revivals and Violence. European Journal of Sociology, 36(01):3, 1995. doi: \texttt{10.1017/S0003975600007098}. URL: \url{http://www.journals.cambridge.org/abstract S0003975600007098}.
		\item[$\bullet$] David Stasavage. Representation and Consent: Why They Arose in Europe and Not Elsewhere. Annual Review of Political Science, 19(1):145-162, 2016. doi: \texttt{10.1146/annurev-polisci-043014-105648}. URL: \url{http://www.annualreviews.org/doi/abs/10.1146/annurev-polisci-043014-105648 http://www.annualreviews.org/doi/10.1146/annurev-polisci-043014-105648}.
		\item[$\bullet$] Gian Vittorio, Claudio Barbaranelli, and Philip Zimbardo. Profiles Personality and Political Parties. Political Psychology, 20(1):175-197, 1999.

	\end{itemize}



\item {\bf Comparative Politics as a Methodology/Subfield}
	\begin{itemize}
		\item[$\bullet$] James Mahoney. Debating the State of Comparative Politics: Views From Qualitative Research. Comparative Political Studies, 40(1):32-38, 2007. doi: \texttt{10.1177/0010414006294816}. URL: \url{http://cps.sagepub.com/cgi/doi/10.1177/0010414006294816}.
		\item[$\bullet$] Gerardo Munck and Richard Snyder. Debating the Direction of Comparative Politics: An Analysis of Leading Journals. Comparative Political Studies, 40(1):5-31, 2007.
		\item[$\bullet$] Gerardo Munck and Jay Verkuilen. Conceptualizing and Measuring Democracy: Evaluating Alternative Indices. Comparative Political Studies, 35(1):5-34, 2002.
		\item[$\bullet$] Barbara Geddes. How the Cases You Choose Affect the Answers You Get: Selection Bias in Comparative Politics. Political Analysis, 2(1):131-150, 1990.
		\item[$\bullet$] Giovanni Capoccia and Daniel Kelemen. The Study of Critical Junctures Theory, Narrative, and Counterfactuals in Historical Institutionalism. World Politics, 59(April):341-369, 2007.
	\end{itemize}
\end{enumerate}


 









%\bibliographystyle{plainnat}
%\bibliography{/Users/hectorbahamonde/RU/Bibliografia_PoliSci/Bahamonde_BibTex2013}

\end{document}