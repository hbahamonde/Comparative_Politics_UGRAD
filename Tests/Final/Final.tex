%This is a LaTeX template for homework assignments
\documentclass{article}
\usepackage{amsmath}
\usepackage{import} % To import email.

\usepackage{multido}
%\newcommand{\cmd}{-x-}
\newcommand{\Repeat}{\multido{\i=1+1}}


\usepackage{geometry}
\geometry{
  %body={6.5in, 8.5in},
  left=0.7in,
  right=0.7in,
  top=0.7in,
  bottom=1in
}


\begin{document}


% Organization 2
\subsection*{For confidentiality, this page is intentionally left blank.}
\clearpage
\newpage


{\centering\section*{Final Examination\\POLC-2300 ``Introduction to Comparative Politics''}}

{\vspace{.5cm}\raggedright{\bf Name}: \line(1,0){200}}. %you can change the length of the lines by changing the number in the curly brackets
{\vspace{.5cm}\hspace{4.5cm}\raggedright{\bf Date}: \line(1,0){100}}. %you can change the length of the lines by changing the number in the curly brackets


{\vspace{.5cm}\raggedright \bf Professor}: Hector Bahamonde.\\
{\bf Class meetings}: \texttt{MWF} 8:00-8:50.\\
{\bf Location}: Stanley Thomas Hall 302.


\vspace{0.5cm}\subsection*{General Instructions}

This exam will take place on {\bf {\input{/Users/hectorbahamonde/RU/Teaching/Comparative_Politics_UGRAD/date_final.txt}\unskip}}. Late tests will not be received, and will be graded with a zero (0). {\bf This exam consists of 4 essays}. This exam is worth 35\% of your grade. The exam has a total of 100 points. The exam is closed-book, covering material from the entire semester (i.e. cumulative). Exam questions have been drawn from the readings, documentaries, lectures, and in-class discussions. Under no circumstances will the following items be accepted: personal notes, books, or written pieces of paper, among other items. The use of electronic devices (smart watches, tablets, laptops, cellphones, and others) are \underline{strictly prohibited}. Violations will be sanctioned with a zero (0). The involved student will be sent to Tulane's Office of Academic Integrity. Write in legible font.  Hard-to-read handwriting will be penalized with points off. Use complete sentences; outlines and lists are not acceptable. Return this handout in full before leaving the classroom. You are free to leave the classroom at any point. However, once you leave, you are not allowed to make changes to this document.
\clearpage
\newpage


\subsection*{Essay Section}

Please, 

\begin{enumerate}
    \item {\bf Answer ONE question from section I} (30\%).
    \item {\bf Answer TWO questions from section II.} (30\% each).
    \item {\bf Answer ONE question from section III.} (10\%).
\end{enumerate}

If you get full credit in every question, you will have 100 points. {\bf Each answer should be at least two pages long, but not longer than three pages}. Answers shorter than two pages long will receive ZERO (0) points. The official minimum is denoted by a ``$\star$'' symbol. Please, answer the question. That is, (1) do not restate the question, (2) avoid giving an indirect answer. Whatever is beyond the third page, will not be read/graded. If there are more than FOUR (4) answers, only the first four will be read/graded.


\paragraph{(I) Methodological Section: \underline{Choose ONE} of the following questions.}


\begin{enumerate}
    \item {\input{/Users/hectorbahamonde/RU/Teaching/Teaching_Questions_Database/3.txt}\unskip} 
    
    \item {\input{/Users/hectorbahamonde/RU/Teaching/Teaching_Questions_Database/4.txt}\unskip} 
\end{enumerate}




\paragraph{(II) Substantive Section: \underline{Answer the next TWO questions}.}


\begin{enumerate}  \setcounter{enumi}{2}
    \item {\input{/Users/hectorbahamonde/RU/Teaching/Teaching_Questions_Database/10.txt}\unskip} 
    \item {\input{/Users/hectorbahamonde/RU/Teaching/Teaching_Questions_Database/9.txt}\unskip} 
\end{enumerate}


\paragraph{(III) Section to Reflect on: \underline{Answer the next question}.}


\begin{enumerate} \setcounter{enumi}{4}
    \item {\input{/Users/hectorbahamonde/RU/Teaching/Teaching_Questions_Database/11.txt}\unskip} 
\end{enumerate}

\clearpage
\newpage


% First Essay
\subsection*{Essay Number \line(1,0){40}.}
\Repeat{24}{\line(1,0){510}\vspace{0.5cm}\\}
\clearpage
\newpage
\Repeat{12}{\hspace{-5mm}\line(1,0){510}\vspace{0.5cm}\\}
\Repeat{13}{\hspace{-5mm}\line(1,0){510}\vspace{0.5cm}\\}
{$\star$}\hspace{8mm}\Repeat{1}{\hspace{-5mm}\line(1,0){498}\vspace{0.5cm}\\}
\Repeat{24}{\line(1,0){510}\vspace{0.5cm}\\}
\clearpage
\newpage



% Second Essay
\subsection*{Essay Number \line(1,0){40}.}
\Repeat{24}{\line(1,0){510}\vspace{0.5cm}\\}
\clearpage
\newpage
\Repeat{12}{\hspace{-5mm}\line(1,0){510}\vspace{0.5cm}\\}
\Repeat{13}{\hspace{-5mm}\line(1,0){510}\vspace{0.5cm}\\}
{$\star$}\hspace{8mm}\Repeat{1}{\hspace{-5mm}\line(1,0){498}\vspace{0.5cm}\\}
\Repeat{24}{\line(1,0){510}\vspace{0.5cm}\\}
\clearpage
\newpage


% Third Essay
\subsection*{Essay Number \line(1,0){40}.}
\Repeat{24}{\line(1,0){510}\vspace{0.5cm}\\}
\clearpage
\newpage
\Repeat{12}{\hspace{-5mm}\line(1,0){510}\vspace{0.5cm}\\}
\Repeat{13}{\hspace{-5mm}\line(1,0){510}\vspace{0.5cm}\\}
{$\star$}\hspace{8mm}\Repeat{1}{\hspace{-5mm}\line(1,0){498}\vspace{0.5cm}\\}
\Repeat{24}{\line(1,0){510}\vspace{0.5cm}\\}
\clearpage
\newpage


% Reflect on Question
\subsection*{Essay Number \line(1,0){40}.}
\Repeat{24}{\line(1,0){510}\vspace{0.5cm}\\}
\clearpage
\newpage
\Repeat{12}{\hspace{-5mm}\line(1,0){510}\vspace{0.5cm}\\}
\Repeat{13}{\hspace{-5mm}\line(1,0){510}\vspace{0.5cm}\\}
{$\star$}\hspace{8mm}\Repeat{1}{\hspace{-5mm}\line(1,0){498}\vspace{0.5cm}\\}
\Repeat{24}{\line(1,0){510}\vspace{0.5cm}\\}
\clearpage
\newpage




\clearpage


% Organization 1
\subsection*{Not intended for answer. This section won't be read/graded. This section is for you to help you organize your answers.}
\clearpage
\newpage

% Organization 2
\subsection*{Not intended for answer. This section won't be read/graded. This section is for you to help you organize your answers.}
\clearpage
\newpage

% credible committments: AR (why they democratize instead of a one time transfer?)

\end{document}