%This is a LaTeX template for homework assignments
\documentclass{article}
\usepackage{amsmath}
\usepackage{import} % To import email.

\usepackage{multido}
%\newcommand{\cmd}{-x-}
\newcommand{\Repeat}{\multido{\i=1+1}}


\usepackage{geometry}
\geometry{
  %body={6.5in, 8.5in},
  left=0.7in,
  right=0.7in,
  top=0.7in,
  bottom=1in
}


\begin{document}

{\centering\section*{Midterm Examination\\POLC-2300 ``Introduction to Comparative Politics''}}

{\vspace{.5cm}\raggedright{\bf Name}: \line(1,0){200}}. %you can change the length of the lines by changing the number in the curly brackets
{\vspace{.5cm}\hspace{4.5cm}\raggedright{\bf Date}: \line(1,0){100}}. %you can change the length of the lines by changing the number in the curly brackets


{\vspace{.5cm}\raggedright \bf Professor}: Hector Bahamonde.\\
{\bf Class meetings}: \texttt{MWF} 8:00-8:50.\\
{\bf Location}: Stanley Thomas Hall 302.


\vspace{0.5cm}\subsection*{General Instructions}

This exam begins at 8:00 and ends at 8:50 AM (CST). Late tests will not be received, and will be graded with a zero (0). This exam consists of 2 essays, and one short question. This exam is worth 25\% of your grade. This exam has a total of 100 points. This exam is closed-book, covering material from the entire semester, up to February 23. Exam questions have been drawn from the readings, lectures and in-class discussions. Under no circumstances will the following items be accepted: personal notes, books, or written pieces of paper, among other items. The use of electronic devices (smart watches, tablets, laptops, cellphones, and others) are \underline{strictly prohibited}. Violations will be sanctioned with a zero (0). The involved student will be sent to Tulane's Office of Academic Integrity. Write in legible font.  Hard-to-read handwriting will be penalized with points off. Use complete sentences; outlines and lists are not acceptable. Return this handout in full before leaving the classroom. You are free to leave the classroom at any point. However, once you leave, you are not allowed to make changes to this document.


\subsection*{Essay Section}

{\bf Please answer TWO (2) of the following questions.} Each essay is worth 45\% of the test. That is, if you get full credit in both, you will have 90 points. {\bf Each answer should be at least one and a half pages long, but not be longer than two pages}. Answers shorter than one and a half pages long will receive ZERO (0) points. The official minimum is denoted by a ``$\star$'' symbol. Please, answer the question. That is, (1) do not restate the question, (2) avoid giving an indirect answer. Whatever is beyond the second page, will not be read/graded. If there are more than TWO (2) answers, only the first two will be read/graded.


\begin{enumerate}
    \item {\input{/Users/hectorbahamonde/RU/Teaching/Teaching_Questions_Database/5.txt}\unskip}
    
    \item {\input{/Users/hectorbahamonde/RU/Teaching/Teaching_Questions_Database/6.txt}\unskip}
    
    
    \item {\input{/Users/hectorbahamonde/RU/Teaching/Teaching_Questions_Database/7.txt}\unskip} 
    
    \item {\input{/Users/hectorbahamonde/RU/Teaching/Teaching_Questions_Database/8.txt}\unskip} 

\end{enumerate}


\subsection*{Short Question}

This question is worth 10\% of the test. That is, if you get the six regimes right, you will have 10 points.

\begin{enumerate}
    \item Following Aristotle's \emph{Politics}, name the three virtuous regimes and their corresponding degenerations.
\end{enumerate}

\clearpage
\newpage


% First Essay
\subsection*{Essay Number \line(1,0){40}.}
\Repeat{24}{\line(1,0){510}\vspace{0.5cm}\\}
\clearpage
\newpage
\Repeat{12}{\hspace{-5mm}\line(1,0){510}\vspace{0.5cm}\\}
{$\star$}\\
\Repeat{12}{\hspace{-5mm}\line(1,0){510}\vspace{0.5cm}\\}
\clearpage
\newpage

% Second Essay
\subsection*{Essay Number \line(1,0){40}.}
\Repeat{24}{\line(1,0){510}\vspace{0.5cm}\\}
\clearpage
\newpage
\Repeat{12}{\hspace{-5mm}\line(1,0){510}\vspace{0.5cm}\\}
{$\star$}\\
\Repeat{12}{\hspace{-5mm}\line(1,0){510}\vspace{0.5cm}\\}
\clearpage
\newpage


% Short question
\subsection*{Short question}

\begin{minipage}{0.45\linewidth}
Virtuous Regimes:\\

 1. \line(1,0){200}\vspace{0.5cm} \\
 2. \line(1,0){200}\vspace{0.5cm} \\
 3. \line(1,0){200}\vspace{0.5cm}

\end{minipage}
 \hspace{\fill}\begin{minipage}{0.45\linewidth}
  \begin{tabular}{ll}
  Corresponding Vicious Version:\\
 \\ 

 1. \line(1,0){200}\vspace{0.5cm} \\
 2. \line(1,0){200}\vspace{0.5cm} \\
 3. \line(1,0){200}\vspace{0.5cm}
  \end{tabular}
\end{minipage}

\clearpage


% Organization 1
\subsection*{Not intended for answer. This section won't be read/graded. This section is for you to help you organize your answers.}
\clearpage
\newpage

% Organization 2
\subsection*{Not intended for answer. This section won't be read/graded. This section is for you to help you organize your answers.}
\clearpage
\newpage

% credible committments: AR (why they democratize instead of a one time transfer?)

\end{document}